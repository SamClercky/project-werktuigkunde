\section{Agenda voor 20-02-17}%
\label{sec:Agenda voor 20-02-17}

\subsection{Verdelen van functies}%
\label{sub:Verdelen van functies}

\textbf{Verkiezing co\"ordinator:} Stemmen voor (geen speciale volgorde):
\begin{itemize}
	\item Andreas: 
	\item Ramses: 
	\item Omar:
	\item Samuel:
	\item Anass:
\end{itemize}

De co\"ordinator moet:
\begin{itemize}
	\item Organiseert werkverdeling
	\item Ziet dat logboek in orde is
	\item Eerlijke verdeling van werklast
	\item Aanspreekpunt voor begeleiders (offici\"ele communicatie)
	\item Redactie eindverslag/presentatie
	\item Neemt deel aan werkzaamheden
	\item Begeleiden van debat en brainstorm momenten
\end{itemize}

\subsection{Organisatorische afspraken}%
\label{sub:Organisatorische afspraken}

\textbf{Zaken die mogelijk besproken moeten worden met assistenten:} 
\begin{itemize}
	\item Moet alle communicatie gaan via het forum op canvas of mogen andere
		media ook benut worden?
	\item Moeten onze berekeningen op canvas staan of mogen we ze ook op GitHub
		zetten? Pro: Version control, Contra: Assistenten hebben minder zicht op 
		wat we doen.
	\item Indien \LaTeX wordt gebruikt voor het verslag, volstaat dan als referentie
		document een bibtex file? Pro: Gemakkelijk inzetbaar, Contra: Soms moeilijk
		leesbaar.
\end{itemize}

\textbf{Gebruikte technologi\"en:}
\begin{itemize}
	\item Verslag in \LaTeX of MS Word
	\item Agenda in \LaTeX of MS Word
	\item Logboek in \LaTeX, MS Word of Markdown
\end{itemize}

\textbf{Invullen van het logboek:} 
\begin{itemize}
	\item Iedereen een eigen logboek laten bijhouden en dan samenvoegen op einde
		van de week
	\item 1 iemand (de logger) vast aanstellen
	\item Elke week iemand anders aanstellen die dan de log doet
\end{itemize}

Invullen van het logboek houd in:
\begin{itemize}
	\item Aanduiden wie/wat/wanneer
	\item Bevat planning van de dag
	\item Controle van de planning (Zie timing and deliverables)
	\item Elke week posten op canvas voor de begeleiders (werkwijze zie pg 13)
\end{itemize}

\textbf{Opstellen van wekelijkse agenda (kan eventueel gecombineerd worden met
logboek):} 
\begin{itemize}
	\item 1 iemand (de verslaggever) vast aanstellen
	\item Elke week iemand anders aanstellen die dan de log doet
\end{itemize}

\Opm Gebruik van \LaTeX vereist dat alle teamleden een werkende installatie hebben
van \LaTeX.

\subsection{Reflecteren op commentaar van verslag eerste dag}%
\label{sub:Reflecteren op commentaar van verslag eerste dag}

\textbf{Dingen die goed waren:}
\begin{itemize}
	\item Goede lengte $<$ 3 pagina's
	\item Begonnen met opgave
	\item Beschrijving idee\"en + specifieke problemen
	\item Ingegaan op het hoe en waarom
	\item Bespreking van methodes
	\item Toevoegen van schema's bij berekeningen
	\item Enkele bronnen werden geciteerd en vermeld in bronvermelding
	\item Beperkt gebruik van we/ons/ik
	\item Beschrijving werkverdeling (misschien net iets te weinig)
	\item Notie van uitgevoerde taken
	\item Reflectie gedane werk
	\item Beschrijving van wat we hebben bijgeleerd
	\item Stil gestaan bij globaal project en een aantal facetten groepsgebeuren
		(voldoende?)
	\item Enkele bevindingen werden gegeven (voldoende?)
\end{itemize}

\textbf{Dingen die beter konden:}
\begin{itemize}
	\item Boekhouding moest totaal aantal uren per teamlid bevatten
	\item Niet iedereen heeft een persoonlijke reflectie (zelfs geen pasfoto)
	\item Niet voldoende ingegaan op de opgegeven probleemstelling
	\item Geen beschrijving van de werking van het product (of niet voldoende)
	\item Geen bespreking van de opstelling + meetinstrumenten
	\item Geen volledige schema's/foto's
	\item Geen vergelijking met bestaande alternatieven
	\item Geen finaal rapport/conclusie of zeer summier
	\item Geen eigen waardeoordeel
	\item Geen +/- voor hoever je bent geslaagd in je opzet
	\item Geen apart document voor referentie werken
	\item Verschillende schrijfwijzen voor bobijn/bobine
	\item Reflectie heeft niet genoeg argumenten?
	\item In de reflectie gebruik van we/ons/ik
\end{itemize}

\subsection{Brainstormen finaal design robot}%
\label{sub:Brainstormen finaal design robot}

\textbf{Beschrijving opdracht:}

Globaal: Een robot die een pad kan volgen, autonoom kabels kan vinden en omhoog
klimmen.

Meer in detail: Ontwerp een volledig autonome robot met Lego Mindstorms en Arduino
die een zwarte lijn kan volgen tot in de buurt van een koord. De regio zal aangegeven
worden met een witte schijf op de grond. De robot zal autonoom het koord moeten 
kunnen vinden, vastnemen en erop klimmen.

\textbf{Wat zijn de technische vereisten (opdracht):}
\begin{itemize}
	\item Robot is maximaal: $200mm\times 200mm\times 200mm$
	\item Kan een zwarte lijn volgen van ducktape
	\item De cirkel: diameter=400mm en H=3mm
	\item Koord: diameter=12mm en H=2000mm
	\item Moet 2m kunnen klimmen
	\item Er mag niet ge\"interageerd worden met de robot (volledig autonoom)
	\item Robot mag de zwarte lijn niet verlaten
	\item Er mogen geen onderdelen worden achtergelaten op het parcour
	\item Tijdslimiet van 10 min
\end{itemize}

\textbf{Beschikbare tools} 
\begin{itemize}
	\item 1 dik koord
	\item Lego mindstorms met 4 motoren
	\item Onder begeleiding 3D-geprinte onderdelen
	\item Extra materiaal mogelijk (elastiekjes, touw, ...)
	\item Extra sensoren mogelijk 
\end{itemize}

\textbf{Onderverdelen in submodules + ploegindeling:} 
\begin{itemize}
	\item 
\end{itemize}

\subsection{Planning van volgende weken}%
\label{sub:Planning van volgende weken}

\textbf{Deadlines:} 
\begin{itemize}
	\item wo 21-02-17: Verschillende concept voorstellen
	\item wo 21-03-03: Werkend prototype
	\item wo 21-03-17 (ten laatste tegen de volgende zondag): Samenvattend eindverslag.
	\item do 20-03-29: Mondelinge presentatie
\end{itemize}
